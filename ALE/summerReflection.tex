% --------------------------------------------------------------
% This is all preamble stuff that you don't have to worry about.
% Head down to where it says "Start here"
% --------------------------------------------------------------
\usepackage{amsmath}
\usepackage{amssymb}
 
\documentclass[12pt]{article}
 
\usepackage[margin=1in]{geometry} 
\usepackage{amsmath,amsthm,amssymb}
 
\newcommand{\N}{\mathbb{N}}
\newcommand{\Z}{\mathbb{Z}}
 
\newenvironment{theorem}[2][Theorem]{\begin{trivlist}
\item[\hskip \labelsep {\bfseries #1}\hskip \labelsep {\bfseries #2.}]}{\end{trivlist}}
\newenvironment{lemma}[2][Lemma]{\begin{trivlist}
\item[\hskip \labelsep {\bfseries #1}\hskip \labelsep {\bfseries #2.}]}{\end{trivlist}}
\newenvironment{exercise}[2][Exercise]{\begin{trivlist}
\item[\hskip \labelsep {\bfseries #1}\hskip \labelsep {\bfseries #2.}]}{\end{trivlist}}
\newenvironment{problem}[2][Problem]{\begin{trivlist}
\item[\hskip \labelsep {\bfseries #1}\hskip \labelsep {\bfseries #2.}]}{\end{trivlist}}
\newenvironment{question}[2][Question]{\begin{trivlist}
\item[\hskip \labelsep {\bfseries #1}\hskip \labelsep {\bfseries #2.}]}{\end{trivlist}}
\newenvironment{corollary}[2][Corollary]{\begin{trivlist}
\item[\hskip \labelsep {\bfseries #1}\hskip \labelsep {\bfseries #2.}]}{\end{trivlist}}

\newenvironment{solution}{\begin{proof}[Solution]}{\end{proof}}
 
\begin{document}
 
% --------------------------------------------------------------
%                         Start here
% --------------------------------------------------------------
 
\title{Summer Reflection}
\author{CJ Girard\\ %replace with your name
CU Presidents Leadership Class Applied Leadership Experience}

\maketitle

\noindent Reflection on Research Experience for Undergraduates (REU) in the University of Colorado Department of Mathematics

\begin{enumerate} 
\item \textit{Something Positive about the role:} This was my first experience working in an academic team towards the same goal as a brilliant postdoc and graduate students. I had a lot to learn, and every day I left feeling like a better version of myself thanks to their human connection, coupled with all they taught me. I loved the feeling of slowly completing my understanding of the big picture over time.
\item \textit{Something Positive About the Org: }Meetings in this group felt challenging constantly - in a sense that we all pushed & motivated each other. I felt pulled along to learn something every moment. I fell behind a lot but I also don't think there are many place were an individual can learn that quickly, and I'd recommend anyone get into research teams with nice, brilliant people willing to teach them. I thrived in that collaborative environment and admire the group for how well they embodied it.
\item \textit{Something Positive about the People: }Each person knew so much and was willing to walk you through whatever you didn't know. One second you can be the student and the next, the teacher. Everyone was united under the ideas which were scary at first, and powerful once you understood them. That feeling of understanding, everyone valued and tried hard to give to each other as much as they sought it for themselves.
\item \textit{Something Aobut your Impact: }I learned graduate-level analysis topics and 2 coding languages that were new to me to uncover what algebra conditions allow for p-operator norms to be presented in spatial tensor products of $\ell^p$ elements and sub-algebras when the method of computation changes. Although I don't understand the full implications for C*-like modules, I helped write code & derive algorithms to generate examples of elements that theoretically may exist according to our group's conjecture.
\item \textit{Lessons Learned: }I left with a lot more wisdom about research. As a student and professional, progress is measured in completed tasks, made deadlines, etc. As a researcher, you ask ten times the questions and answer one tenth. Executive function can only take you so far when it's not obvious what to explore or do next. If you're not careful, the narrative of 'what are you doing ... why is it useful' pops up in the mind where the passion and drive forward should be. Work is less straightforward than a problem set, so it can be hard to get validation as one might in other roles. I grew in the degree to which I could feel a sense of pride even when I wasn't sure where I was going. Everyone in any area can benefit from feeling confident despite feeling lost. Trust yourself to eventually find the way with fortitude. 
\item \textit{Challenge: }I have trouble asking for help. And I had to ask many times for people to slow down and help me catch up on the concepts. Face-to-face teaching moments are so much more powerful that a textbook or video lecture series. But honest communication from the student about their needs means they don't have to wait for exactly what they need from an instructor. Being honest and being vulnerable was difficult this Summer, but we're strongest when we dare to show our weakness. So please ask for help, students.
\end{enumerate}
 
% --------------------------------------------------------------
%     You don't have to mess with anything below this line.
% --------------------------------------------------------------
 
\end{document}
